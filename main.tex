\documentclass{article}
\usepackage[utf8]{inputenc}
\usepackage[
    type={CC},
    modifier={by-sa},
    version={4.0},
]{doclicense}

\usepackage{amsmath}
\usepackage{amssymb}
\usepackage{amsthm}
\usepackage{amsfonts}
\usepackage{tikz-cd}
\usepackage{marginnote}
\usepackage[all]{xy}


\newtheorem{theorem}{Theorem}[subsection]
\newtheorem{proposition}[theorem]{Proposition}
\newtheorem{lemma}[theorem]{Lemma}
\newtheorem{corollary}[theorem]{Corollary}
\newtheorem{example}[theorem]{Example}
\newtheorem{definition}[theorem]{Definition}
\newtheorem{remark}[theorem]{Remark}

\newcommand{\gr}{{\rm gr ~}} %associated graded
\newcommand{\acts}{\curvearrowright} % G \acts X
\newcommand{\del}{\partial}
\newcommand{\pt}{{\rm pt ~}}
\newcommand{\rk}{{\rm rank ~}}

\newcommand{\End}{{\rm End }}
\newcommand{\Hom}{{\rm Hom }}
\newcommand{\Sym}{{\rm Sym }}
\newcommand{\Spec}{{\rm Spec }}
\newcommand{\Proj}{{\rm Proj }}
\newcommand{\Span}{{\rm Span}}
\newcommand{\Rep}{{\rm Rep ~}}
\newcommand{\Supp}{{\rm Supp ~}}

% mathbb 
\newcommand{\C}{\mathbb{C}}
\newcommand{\Z}{\mathbb{Z}}

% mathfrak 
\newcommand{\ff}{\mathfrak{f}}
\renewcommand{\k}{\mathfrak{k}}
\newcommand{\g}{\mathfrak{g}}
\renewcommand{\tt}{{\mathfrak{t}}}
\renewcommand{\sl}{\mathfrak{sl}}
\newcommand{\m}{\mathfrak{m}}

% mathcal
\newcommand{\U}{\mathcal{U}}
\newcommand{\OO}{{\mathcal{O}}}
\newcommand{\M}{\mathcal{M}}
\newcommand{\fl}{\mathcal{F}l}
\newcommand{\G}{\mathcal{G}}
\newcommand{\W}{\mathcal{W}}

% misc
\newcommand{\inv}[2][1]{{#2}^{-#1}}
\newcommand{\st}{\,\big|\,}


\title{WARTHOG notes}
\author{Ben Webster}


\begin{document}

\maketitle

Note to contributors: this is intended to be a collaboratively written set of notes for the WARTHOG workshop help in August 2017.  By editing it, you are agreeing to the Creative Commons license below, and that in particular, this work can be published in an appropriate forum.  All contributors we can identify will be credited in any version of these notes we distribute.

\newpage 
\section{Monday: Ben}

\begin{definition} 
an \underline{almost commutative ring} $R$ is a ring with a filtration 
\[
    A_0 \subset A_1 \subset \cdots     
\]
such that 
\[
    A_i A_j \subset A_{i+ j} \textrm{ and for }   [A_i , A_j] \subset A_{i+j - s}    
\]
for some positive integer $s$.
\end{definition}

We can take the commutative graded of such a ring
\[
    \gr(A) = \bigoplus A_j / A_{j-1}    
\]
to obtain a \textit{semiclassical} conical Poisson ring.  

\begin{definition} 
a \underline{conical Poisson ring} has
\begin{itemize}
    \item a $\Z_{\ge 0}$-grading, and 
    \item a second \textit{homogeneous} operation $\{-,-\}: R\times R \rightarrow R$ of degree $-s$ for which 
    \begin{itemize}
        \item Lie bracket, and 
        \item Liebniz
    \end{itemize} relations hold.
\end{itemize} 
\end{definition}

\textbf{Think} the functor $C \mapsto \{ - , C\}$ from functions to vector fields. In particular the Jacobi identity implies it is a Lie algebra homomorphism. 

\begin{remark} 
There is a classical limit functor from filtered to graded [conical Poisson] rings 
\[
    A \mapsto (\gr A, \{-,-\})
\] 
such that $\{\overline{a},\overline{b}\} = \overline{[a,b]}$. 
\end{remark}

\begin{example} 
% \begin{math}\C \Z \del \end{math}
Let $A = \mathbb{C}\langle x,d/dx\rangle$ be the ring of polynomial differential operators. Then the filtered pieces are 
\begin{itemize}
    \item $A_1 = \Span(1,x,d/dx)$
    \item $A_n = A_1^n$
\end{itemize} 
and 
\begin{itemize}
    \item $\gr A = \C [x,p]$
    \item $\{ f,g \} = \frac{\del f}{\del p} \frac{\del g}{\del x} - \frac{\del g}{\del p}\frac{\del f}{\del x}$
    \item $\{ p,x\} = 1$.
\end{itemize}
Note, the symbol $p$ denotes the image of the operator $\del/\del x$ in $A_1/A_0$. 
\end{example}

\begin{example} 
The canonical map $\U\g\rightarrow \C[\g^\ast]$ equips $\C[\g^\ast]$ with KKS Poisson structure got from the bracket $[X,Y] = XY - YX$ on $\U\g$. Note $(\U\g)_1 = \C\oplus \g$. 
\end{example}



\newpage

\section{Hypertoric Enveloping Algebra. Part I (Nick)}

\subsection{Weyl algebra}

Once and for all, fix a positive integer $n\ge1$. Consider the Weyl algebra
\begin{equation}
\begin{array}{cc}
    D = \mathbb{C}\langle x_1,\partial_1, x_2, \partial_2, \dots, x_n, \partial_n \rangle = \\ \{\textrm{polynomial differential operators on  } \mathbb{C}[x_1,x_2, \dots, x_n]\}
\end{array}
\end{equation}

As usual, the algebra $D$ has a filtration given by the order of the differential operators. However, we prefer to consider a different filtration, adjusted to the fact that we are considering differential operators on the affine space $\mathbb{C}^n$. Specifically, for $k\ge0$ let
\begin{equation}
F_k D = \Big\{ p_{k}(x_1,\partial_1, \dots, x_n, \partial_n)\Big| ~ p_{k} \textrm{ is a polynomial of degree at most } k \Big\}.
\end{equation}
\begin{example} 
$x_1\partial_1\in F_2 D$, $1\in F_0D$, $x_1^3\partial_2^2\in F_5D$.
\end{example}
Note that the associated graded of the filtration is the polynomial ring:
\begin{equation}
\displaystyle \gr D = \bigoplus_{k\ge0} F_k D/{F_{k-1}D} \cong \mathbb{C} [x_1,y_1,\dots, x_n, y_n],
\end{equation}
where $x_i$ denotes (the image of) multiplication by $x_i$, and $y_i$ denotes (the image of) $\partial_i$.

Consider the action $(\mathbb{C}^*)^n \acts D$ given by $t\cdot x_i = t_ix_i$, $t\cdot \partial_i = t_i^{-1}\partial_i$ for $t=(t_1,\dots,t_n)\in (\mathbb{C}^*)^n$.

For any $z\in \mathbb{Z}^n$, define the weight space $D_z=\{a\in D | t\cdot a = t^z a, \forall t \in (\mathbb{C}^*)^n\}$, where we use notation $(t_1,\dots,t_n)^{(z_1,\dots, z_n)} = t_1^{z_1}\dots t_n^{z_n}$. As usual, one has the decomposition $D=\bigoplus_{z\in\mathbb{Z}^n}D_z$.

\begin{example}
$D_0$ contains the elements $x_i\partial_i$ and $\partial_ix_i$, for any $i$. There elements are going to be of extreme importance for us. So let us fix notations for them. Let $h_i^+=x_i\partial_i$, $h_i^-=\partial_ix_i$. 
\end{example}
Note that $h_i^-=h_i^++1$. So, these two elements carry more or less the same information. At this point, it looks like it would be more convenient to switch $+$ with $-$ in the notation. However, this particular choice of notation will become convenient later on.

Note that any polynomial expression of $h_i^+$'s (as well as $h_i^-$'s) will represent an element of $D_0$. So, it is immediate that
$$
D_0 \supset \mathbb{C}[h_1^+,\dots,h_n^+] = \mathbb{C}[h_1^-,\dots, h_n^-].
$$
What is less obvious is that in fact the inclusion above is an equality. We are going to prove it next.

To prove that, we need to understand how to write down an expression of the sort $x_1\partial_1^3x_1x_2^4\partial_2^4 \in D_0$ as a polynomial applied to $(h_1^+,\dots, h_n^+)$.

\begin{lemma}\label{lm:action_of_h_on_D_0}
If $a\in D_0$, then $[h_i^\pm,a]=z_ia$. 
\end{lemma}
Note that it does not matter, which of $h_i^+$, $h_i^-$ we use in the formula, because they differ by a constant.
\begin{proof}

\end{proof}

\begin{lemma}\label{lm:express_as_prod_of_hs}
For any $k\ge0$ one has,
$$
\partial_i^kx_i^k=(h_i^+ + 1)(h_i^+ + 2)\dots (h_i^+ + k),
$$
$$
x_i^k\partial_i^k=(h_i^- - 1)(h_i^- - 2)\dots (h_i^- - k).
$$
\end{lemma}
\begin{proof}

\end{proof}

Lemma \ref{lm:express_as_prod_of_hs} gives an efficient way of expressing any element in $D_0$ as a polynomial in $h_i^\pm$'s.
For instance,

\begin{equation*}
\begin{array}{cc}
     x_1\partial_1^3x_1x_2^4\partial_2^4 = (x_1^2\partial_1^2)(\partial_1x_1)(x_2^4\partial_2^4)= \\
          =
    (h_1^--1)(h_1^--2)(h_1^++1)(h_2^--1)(h_2^--2)(h_2^--3)(h_2^--4).
\end{array}
\end{equation*}

\begin{corollary}
$D_0 = \mathbb{C}[h_1^+,\dots,h_n^+] = \mathbb{C}[h_1^-,\dots, h_n^-].$
\end{corollary}

Now that we have described $D_0$, let us discuss how to describe $D_z$ for $z\not=0$.

For any $z\in \mathbb{Z}^n$, denote by $z_+$ and $z_-$ the vectors with coordinates $(z_+)_i=\max(z_i,0)$, $(z_-)_i=\max(-z_i,0)$. Note that $z=z_+-z_-$. Denote $m^z=x^{z_+}\partial^{z_-}\in D_z$.

\begin{example}
If $z=(5,-2,0,1,1)$, then 
\[
m^z=x^{(5,0,0,1,1)}\partial^{(0,2,0,0,0)}=
x_1^5x_4x_5\partial_2^2\,.
\]
\end{example}

For any $f(h)=f(h_1^+, \dots, h_n^+)\in D_0$ and any $z\in \mathbb{Z}^n$, let $$f_z(h) = f(h_1^++z_1,\dots,h_n^++z_n).$$

\begin{lemma}\label{lm:commuting_poly_and_m}
$f(h)m^z = m^z f_z(h)$
\end{lemma}
\begin{proof}

\end{proof}

\begin{corollary} \label{cor:description_of_D_z}
For any $z\in \mathbb{Z}^n$, $D_z=\mathbb{C}[h_1^\pm,\dots,h_n^\pm]\cdot m^z$.
\end{corollary}
\begin{proof}
Follows from Lemmas \ref{lm:express_as_prod_of_hs}, \ref{lm:commuting_poly_and_m}.
\end{proof}

\begin{example}
Let $n=1$. Let us represent the element $x_1^8\partial_1^5\in D_3$ in the form described in Corollary \ref{cor:description_of_D_z}:

\begin{equation*}
    \begin{array}{ll}
         x_1^8\partial_1^5 = x_1^3(x_1^5\partial_1^5) &= 
         x_1^3(h_1^--1)(h_1^--2)\dots(h_1^--5) =\\
         &=(h_1^--4)(h_1^--5)\dots(h_1^--8)x_1^3.
    \end{array}
\end{equation*}
\end{example}

Corollary \ref{cor:description_of_D_z} provides the following description of $D$ (as a vector space):
$$
D=\bigoplus_{z\in\mathbb{Z}^n}D_z = \bigoplus_{z\in\mathbb{Z}^n}\mathbb{C}[h_1^\pm,\dots,h_n^\pm]\cdot m^z
$$
To understand the algebra structure of $D$ in terms of this decomposition, we need a formula for computing the product $f(h)m^z \cdot g(h)m^w$, for polynomials $f,g$ and $z,w\in\mathbb{Z}^n$. Lemma \ref{lm:express_as_prod_of_hs} guarantees that
$$
f(h)m^z \cdot g(h)m^w = f(h)g_z(h)m^z\cdot m^w,
$$
so reduces the problem to computing the product $m^z\cdot m^w$. Note that the latter apriori belongs to $D_{z+w}$, however there is no reason to expect it to be $m^{z+w}$. In fact, the following is true.

\begin{proposition} Let 

\begin{equation*}
    \begin{array}{ccc}
         [h_i]^k=&\left\{
         \begin{array}{lll}
              1, & \textrm{if } k=0,\\
              x_i^k\partial_i^k, & \textrm{if } k>0,\\
              \partial^{-k}x_i^{-k}, & \textrm{if } k<0,
         \end{array}
         \right.
         &
         =\left\{
         \begin{array}{lll}
              1, & \textrm{if } k=0,\\
              (h_i^+ + 1)\dots (h_i^+ + k), & \textrm{if } k>0,\\
              (h_i^- + 1)\dots (h_i^- + |k|), & \textrm{if } k<0.
         \end{array}
         \right.
    \end{array}
\end{equation*}

Then 

$$
    m^z\cdot m^w = \left(\prod_{\substack{z_iw_i<0 \\ |z_i|\leq |w_i|}}[h_i]^{z_i}\right)m^{z+w}\left(\prod_{\substack{z_iw_i<0 \\ |z_i|> |w_i|}}[h_i]^{w_i}\right).
$$
\end{proposition}
\begin{proof}
By example (in exercises).
\end{proof}

Note that each $z_i$ has the same sign as $w_i$, then the naive expectation will hold: $m^z\cdot m^w=m^{z+w}$.

\subsection{Hypertoric Enveloping Algebra}

To construct hypertoric enveloping algebra, we need to fix a subtorus $K\subset (\mathbb{C^*}^n)$. Throughout the notes we will assume the following genericity section:
\begin{align}\tag{S}\label{subtorus_condition}
    K \textrm{ does not contain any coordinate subtorus.}
\end{align}

Let $T=(\mathbb{C^*})^n/K$. Consider the following short exact sequences:

\begin{tikzcd}
    1\arrow{r} & K \arrow{r}& (\mathbb{C^*})^n \arrow{r}& T\arrow{r} & 1 \\ 
    0\arrow{r} & \mathfrak{k} \arrow{r}& \mathbb{C}^n \arrow{r}& \mathfrak{t}\arrow{r} & 0 \\
    0 & \arrow{l}\mathfrak{k} & \arrow{l}(\mathbb{C}^n)^* & \arrow{l}\mathfrak{t}^*& \arrow{l} 0 \\
    0 & \arrow{l}\mathfrak{k}_{\mathbb{Z}}\ar[equal]{d}& \arrow{l}\mathbb{Z}^n \ar[equal]{d}& \arrow{l}\mathfrak{t}^*_{\mathbb{Z}}\ar[equal]{d}& \arrow{l} 0 \\
     & \Hom(K,\mathbb{C}^*) & \Hom((\mathbb{C}^*)^n,\mathbb{C}^*) & \Hom(T,\mathbb{C}^*) &  \\
\end{tikzcd}

\begin{definition}
The hypertoric enveloping algebra (HEA, for short) is
$$
U=D^K = \bigoplus_{z\in \mathfrak{t}^*_{\mathbb{Z}}\subset\mathbb{Z}^n} \mathbb{C}[h_1^\pm,\dots,h_n^\pm]\cdot m^z.
$$
\end{definition}

\begin{example}\label{ex:diagonal_in_C3}
Let $n=3$, $K=\mathbb{C}^*_\Delta$ be the diagonal copy of $\mathbb{C}^*$ sitting inside $(\mathbb{C}^*)^3$. Then $T=(\mathbb{C}^*)^3/\mathbb{C}^*_\Delta\cong (\mathbb{C}^*)^2$. We have short exact sequences:

\begin{tikzcd}[row sep=tiny]
  1\arrow{r} & \mathbb{C}^* \arrow{r}& (\mathbb{C^*})^3 \arrow{r}& (\mathbb{C^*})^2\arrow{r} & 1 \\ 
  & t \ar[maps to]{r}&(t,t,t)&&\\
  &&(t_1,t_2,t_3)\ar[maps to]{r}& (t_1/t_3,t_2/t_3)&
\end{tikzcd}
\todo{There is a commutative diagram in the source here, which does not compile.}


% \begin{tikzcd}[row sep=large]
%     1\arrow{r} & \mathbb{C} \arrow{r}{\begin{bmatrix} 1  \\ 1 \\ 1 \end{bmatrix}} & [3em]\mathbb{C}^3 \arrow{r}{\begin{bmatrix} 1&0&-1  \\ 0&1&-1 \\ \end{bmatrix}}&[3em] \mathbb{C}^2\arrow{r} & 1 \\ 
%     1 & \arrow{l}\mathbb{C} & \arrow{l}[swap]{\begin{bmatrix} 1&1&1 \end{bmatrix}}\mathbb{C}^3 & \arrow{l}[swap]{\begin{bmatrix} 1&0\\0&1 \\ -1&-1 \\ \end{bmatrix}} \mathbb{C}^2 &\arrow{l} 1 \\
%     1 & \arrow{l}\mathbb{Z} & \arrow{l}\mathbb{Z}^3 &  \arrow{l}\mathbb{Z}^2 &\arrow{l} 1 
%   \end{tikzcd}

\todo{picture here?}
\end{example}

The last question we will address in this section is: What is the center of $U$?

Recall that Lemma \ref{lm:action_of_h_on_D_0}, if $a\in D_0$, then $[h_i^+,a]=z_ia$. In particular, we get that $Z(U)\subset D_0$.

Also, recall that by Lemma \ref{lm:commuting_poly_and_m} $f(h)m^z=m^zf_z(h)$. So, typically $Z(U)$ is strictly smaller than $D_0$.  Moreover $f(h)$ is central iff $f(h)=f_z(h)$, for all $z\in\mathfrak{t}^*_{\mathbb{Z}}\subset \mathbb{Z}^n$.

\begin{proposition}\label{prop:center_of_HEA}
$Z(U)=\Sym(\mathfrak{k})\subset \Sym(\mathbb{C}^n)=\mathbb{C}[h_1^+,\dots,h_n^+].$
\end{proposition}
\begin{proof}

\end{proof}

\begin{example}
In the setting of Example \ref{ex:diagonal_in_C3}, $Z(U)=\mathbb{C}[h_1^++h_2^++h_3^+]$.
\end{example}

%%% Local Variables:
%%% TeX-master: "main"
%%% End:

\input{monday3}
\newpage

\section{Weight Modules (Ben)}

Recall that $D$ stands for the Weyl algebra of polynomial differential operators on $\mathbb{C}[x_1,\dots,x_n]$. If $K$ is a subtorus of $(\mathbb{C^*})^n$ (not containing any coordinate subtorus), then we define the hypertoric enveloping algebra (HEA) as $U=D^K$. The HEA contains an abelian subalgebra $D_0=\mathbb{C}[h_1^+,\dots,h_n^+$.

In this section we will describe simple weight modules of $U$. Throughout the section, by a $U$-module we mean a left $U$-module.

Let $M$ be a finite dimensional $U$-module. Let us decompose this module into eigenspaces of $h_i^+$'s.

\begin{definition}
Let $\chi$ be a surjective homomorphism $D_0\to \mathbb{C}$. Denote by $m_\chi$ the maximal ideal $\ker(\chi:D_0\to \mathbb{C})$. Define
$$
M_\chi = \{v\in M | m_\chi^N=0 \textrm{ for some } N\ge1\}.
$$
Alternatively, if we denote $\chi_i=\chi(h_i^+)$, then
$$
M_\chi = \{v\in M | (h_i^+-\chi_i)^N=0 \textrm{ for some } N\ge1\}.
$$
\end{definition}
\begin{definition}
A $U$-module $M$ is called a weight module if $\displaystyle M\cong \bigoplus_{\chi}M_\chi$.
\end{definition}


\begin{theorem}
If $M$ is a finite dimensional $U$-module, then it is a weight module.
\end{theorem}

\begin{proof}
For any $i$, consider the Jordan decomposition $h_i^+=h_i^{ss}+h_i^{n}$, where $h_i^{ss}$ acts on $M$ semi-simply and $h_i^n$ nilpotently. Recall that both $h_i^{ss}$ and $h_i^n$ are polynomials in $h_i^+$. In particular the operators $h_i^+, 1\le i\le n$ are simultaneously diagonalizable.
\end{proof}

\begin{remark}
The action of $h_i^n$ is a $U$-module homomorphism $M\to M$, and moreover it commutes with any homomorphism of $U$-modules $M\to N$.
\end{remark}

\begin{corollary}
If $L$ is a simple finite dimensional $U$-module, then each $h_i$ acts semi-simply.
\end{corollary}

\begin{proof}
The image of $h_i^n$ is a submodule, and it cannot be the whole $L$. So, $h_i^n$ acts trivially.
\end{proof}

For any (surjective homomorphism) $\chi:D_0\to \mathbb{C}$, define the module $U$-module $P^{(1)}_\chi=U/Um_\chi$.
\marginnote{do we need modules $P_\chi^{(N)}$?}
It is an easy observation that for any $U$-module $M$, one has $M_\chi=\Hom(P_\chi^{(1)}, M)$. In particular, if $L$ is a simple $U$-module  and $L_\chi\neq \emptyset$ for some $\chi$, then $L$ is a quotient of $P_\chi^{(1)}$.

Let $w \in P_\chi^{(1)}$ be image of $1$. Note that $w\in (P_\chi^{(1)})_\chi $.

\begin{theorem}
$P_\chi^{(1)}$ has a basis given by $\{m^\lambda \cdot w\}_{\lambda\in \mathfrak{t}^*_{\mathbb{Z}}}$.
\end{theorem}
\begin{proof}

\end{proof}

\begin{corollary}
If $m_\chi,m_\nu\in \Spec_m(D_0)$, then $\left(P_\chi^{(1)}\right)_\nu=\left\{\begin{matrix}1&,\chi-\nu \in \mathfrak{t}^*_\mathbb{Z},\\0&,\textrm{otherwise}. \end{matrix}\right.$ 
\end{corollary}

\begin{lemma}\label{lm:proper_submodule_criterion}
A $U$-submodule $N\subset P_\chi^{(1)}$ is proper iff $N_\chi=0$.
\end{lemma}

\begin{proof}
If $N_\chi\neq\emptyset$, then $N_\chi=\left(P_\chi^{(1)}\right)_\chi\ni w$, so $N=P_\chi^{(1)}$.
\end{proof}

\begin{theorem}
Each $P_\chi^{(1)}$ has a unique simple quotient.
\end{theorem}
\begin{proof}
Existence is guaranteed by Zorn's Lemma. Uniqueness follows from the fact that sum of two proper submodules of $P_\chi^{(1)}$ is again a proper submodule. The latter fact follows direcly from Lemma \ref{lm:proper_submodule_criterion}.
\end{proof}

Denote the unique simple quotient of $P_\chi^{(1)}$ by $S_\chi$.

\begin{remark}
The module $S_\chi$ is not necessarily finite dimensional, but it is a weight module, since $P_\chi^{(1)}$ is.
\end{remark}

\begin{proposition}
For any $m_\chi,m_\nu\in\Spec_m(D_0)$, the following are equivalent:
\begin{enumerate}
    \item $P_\chi^{(1)}\cong P_\nu^{(1)}$,
    \item $S_\chi\cong S_\nu$,
    \item $\left(S_\chi\right)_\nu \neq \emptyset$,
    \item $\dim \left(S_\chi\right)_\nu =1 $,
    \item $m^{\chi-\nu}m^{\nu-\chi}w\neq 0\in P_\chi^{(1)}$.
\end{enumerate}
\end{proposition}

\begin{proof}

\end{proof}

Let us define an equivalence relation on the space of characters $D_0\to \mathbb{C}$, by declaring $\chi \sim \nu$, if any of the five conditions above holds.

\begin{theorem}\label{thm:equivalence_of_simples}
Let $m_\chi,m_\nu\in\Spec_m(D_0)$. Then $\chi \sim \nu$ iff $\chi-\nu\in\mathfrak{t}^*_\mathbb{Z}$ and there is no $i$ such that $h_i^-(\chi)$ and $h_i^+(\nu)$ both are integers and have opposite signs.
\end{theorem}

\begin{example}
Let $n=2$, $K=\{1\}$, $\mathfrak{t}^*_\mathbb{Z}\cong \mathbb{Z}^2$. Get $4$ equivalence classes on $\mathfrak{t}^*_\mathbb{Z}$. \marginnote{need a picture here}
\end{example}

Fix $\xi \in \mathfrak{k}^*_\mathbb{Z}$, and look at the (lattice in) affine space $\xi+\mathfrak{t}^*_\mathbb{Z}$. Each function $h_i^\pm$ is integral either at all points of $\xi+\mathfrak{t}^*_\mathbb{Z}$, or at none of them. 

If each $h_i^\pm$ is integral at none of the points $\xi+\mathfrak{t}^*_\mathbb{Z}$, then the condition in Theorem \ref{thm:equivalence_of_simples} vacuously holds true, and so all the points in $\xi+\mathfrak{t}^*_\mathbb{Z}$ are equivalent. In particular, for a generic $\xi$, there is unique simple module with a non-zero weight space in $\xi+\mathfrak{t}^*_\mathbb{Z}$. This is given by $P_\chi^{(1)}$ for any $\chi \in \xi+\mathfrak{t}^*_\mathbb{Z}$.

If some $h_i^\pm$ are integral at all the points of $\xi+\mathfrak{t}^*_\mathbb{Z}$, then this affine lattice will be divided by the level sets $h_i^\pm=0$ into several chambers. The set of chambers containing at least one point will be in a bijection with the set of simple $U$-modules. Moreover, the chambers with finitely many points will correspond exactly to the finite dimensional $U$-modules.


\newpage 

\section{Tuesday: Justin}

\subsection{Central quotients of the hypertoric enveloping algebra}

Recall the exact sequence of groups (in Justin's versus Nick's notation) 
\[
    \xymatrix{1 \ar[r] & G \ar[r]^-{i} & (\C^\times)^n \ar[r]^-{f} & F \ar[r] & 1 } 
\]
can be differentiated 
\[
    \xymatrix{0 \ar[r] & \g \ar[r]^-{T_ei} & \C^n \ar[r]^-{T_e f} & \ff \ar[r] & 0 } 
\]
and dualized 
\[
    \xymatrix{0 & \g^\vee \ar[l] & (\C^n)^\vee \ar[l] & \ff^\vee \ar[l] & 0 \ar[l] }\,.
\]
We have shown that the lattice points $\ff^\vee_\Z$ index the decomposition of $D^G$ as follows
\[
    U = D^G = \bigoplus_{z \in \ff_\Z^\vee} D_0 \cdot m^z \qquad D_0 = \C[h_1^\pm \ldots h_n^\pm] \qquad h_i^- = h_i^+ + 1\,.
\]

\textbf{Note.} The associated graded $\gr D_0 \cong \Sym (\C^n) = \C[(\C^n)^\vee]$ is \textit{canonically} isomorphic to the symmetric algebra of the Cartan. You can get $D_0\cong \C[(\C^n)^\vee]$ if you privilege $h_i^+$ (over $h_i^-$). 

\textbf{Proposition.} (Nick's talk.) 
\[
\xymatrix{
    \C[\g^\vee] \ar[r] \ar@{=}[d] & \C[(\C^n)^\vee] \ar@{=}[d] \\
    \Sym(\g) \ar[r]^-{i_\ast} & \Sym(\C^n) \\
    Z(U) \ar@{-->}[u]^{\cong} \ar[r] & D_0\ar[u]^{\cong} 
}    
\]
Nick claimed the existence of the dashed arrow. Justin will prove it. 

First, would-be consequences. 

\textbf{Consequences.} TFAE 
\begin{enumerate}
    \item $m_\lambda \subset Z(U)$ are maximal ideals
    \item points $\lambda\in \g^\vee$ 
    \item cosets $\Lambda_\C\subset (\C^n)^\vee$ for $\ff^\vee$-action  
\end{enumerate}

\textbf{Definition.} \[ U_\lambda = U/U\cdot m_\lambda \] is a central quotient of $U$. Note $U\cdot m_\lambda$ is a 2-sided ideal, not to be confused with Ben's $\m_\lambda$ which kills less stuff. 

\textbf{Recall:} any indecomposable [?] module $M$ must have its support (a coherent sheaf valued in Set) $\Supp(M)\subset (\C^n)^\vee$ contained in a single coset for $\ff_\Z^\ast$. Note, the support is equal to the set of nonzero weights for $M$.

\textbf{Note:} Even if $\Supp(M) \subset \Lambda_\C$ it does not have to be true that $M$ is a $U_\Lambda$-module. 

\textbf{Proposition.} TFAE
\begin{enumerate}
    \item $M$ is a $U_\lambda$-module 
    \item $M = (M)^{m_\lambda} = \{m \in M \st m_\lambda \cdot m = 0\}$. 
    \item the scheme-y support of $M$ is also contained in a single coset $s\Supp(M)\subset\Lambda_\C$ 
\end{enumerate}

\textbf{Example} 
Consider the following exact sequence of groups
\[
    \xymatrix{\C^\times \ar[r]^-{\Delta} & (\C^\times)^2 \ar@{->>}[r] & \C^\times = F \\ & (t_1,t_2) \ar@{|->}[r] & (t_2/t_1) }    
\]
it differentiates to 
\[
    \xymatrix{
    \g = \C \ar[r]^{\begin{bmatrix}1 \\ 1\end{bmatrix}} & \C^2 \ar[r]^{\begin{bmatrix}-1 & 1\end{bmatrix}} & \C = \ff
    }    
\]
whence $Z(U) = \C[h_1^+ + h_2^+]\subset \C[h_1^+,h_2^+]\cong D_0$. 
This situation has for $\lambda = 0$ for instance $M$ a $U_\lambda$-module $\iff$ $(h_1^+ + h_2^+)m = 0$ for all $m \in M$. 

\textbf{Warning!} The modules we talk about are always (?) generalized. 

\textbf{Definition.} (Should have come before.) Call a $U-$ (or $U_\lambda$-) module \underline{weight} if 
\[ M = \bigoplus M_z \] and all $M_z$ are finite dimensional generalized weight spaces for $D_0$. 

There is a $[D,U_\lambda]$-bimodule
\[B_\lambda = D/D\cdot m_\lambda\] that allows us to go back and forth between $D$-mod and $U_\lambda$-mod via 
\[\xymatrix{
    D\textrm{-mod}\ar[r]^{\hom(B_\lambda, - )\cong (-)^{m_\lambda}} & U_\lambda\textrm{-mod} \ar[l]^{fill}
}\]



\newpage
\section{Quiver presentation of modules over HEA} % Ben

In the previous section, we defined category $\mathcal{O}$ of modules over $U_\eta$. Let us discuss what do these modules look like. 

\begin{example}
Let $\alpha\in\{+,-\}^n$ be a feasible sign vector. Then the the module $S_\alpha$ belongs to category $\mathcal{O}$ iff $\alpha$ is $\xi$-bounded.
\end{example}

To make further analysis, let us consider $D$-modules first. Moreover, it will be instructive to look at the case $n=1$ first, i.e. $D=\langle x, \partial\rangle$. What does a weight module over $D$ with integer weights look like?

Let $M$ be a $D$-module, such that $M=\bigoplus_{k\in \mathbb{Z}} M_k$, where the weights are meant with respect to $h^+$-action. The elements $x$ and $\partial$ will act in the following fashion:

\begin{equation}
\xymatrix{
\dots~~ \ar@/^1.0pc/@[black][r]^{x}
  & M_{-2}   \ar@/^1.0pc/@[black][l]^{\partial} \ar@/^1.0pc/@[black][r]^{x}
  & M_{-1}  \ar@/^1.0pc/@[black][l]^{\partial} \ar@/^1.0pc/@[black][r]^{x}
  & M_0 \ar@/^1.0pc/@[black][l]^{\partial}
  \ar@/^1.0pc/@[black][r]^{x}
  & M_1 \ar@/^1.0pc/@[black][l]^{\partial}
  \ar@/^1.0pc/@[black][r]^{x} 
  & ~~\dots \ar@/^1.0pc/@[black][l]^{\partial}
  }
\end{equation}

An important question is when the compositions $$\xymatrix{M_n\ar[r]^{x}& M_{n+1}\ar[r]^{\partial}& M_{n}}$$ and $$\xymatrix{M_n\ar[r]^{\partial}& M_{n-1}\ar[r]^{x}& M_{n}}$$ are isomorphisms. By definition of $h^\pm$, those compositions are equal to actions of $h^-$ and $h^+$, respectively. That means that $x\partial:{M_n\to M_{n}}$ is a nilpotent map when $n=-1$ and isomorphism otherwise. Likewise the map $\partial x:{M_n\to M_{n}}$ is nilpotent when $n=0$ and isomorphism otherwise. Summarizing, the weight spaces $M_k$, $k\le-1$ are all isomorphic to $M_{-1}$, as well as the weight spaces $M_k$, $k\ge0$ are all isomorphic $M_0$. A nontrivial piece of information carried by module $M$ is how the maps $x:M_{-1}\to M_0$ and $\partial:M_0\to M_{-1}$ assempbly to a nilpotent representation of the quiver
$$
\Gamma = \xymatrix{\cdot \ar@/^1.0pc/@[black][r]
  & \cdot  \ar@/^1.0pc/@[black][l]}
$$
Intuitively, it is clear that have a nilpotent representation of $\Gamma$, one can reconstruct the $U$-module $M$. This idea can be made rigorous.

Denote by $D-mod_\mathbb{Z}$ the category of weight $D$-modules with integral weights. Denote by $NilRep(\Gamma)$ the category of nilpotent representations of the graph $\Gamma$.
\begin{theorem}
The functor $D-mod_\mathbb{Z}\to NilRep(\Gamma)$ given by $$  \xymatrix{M\ar@{|->}[r]&M_{-1} \ar@/^1.0pc/@[black][r]^{x} 
  & M_0  \ar@/^1.0pc/@[black][l]^{\partial}}$$ is an equivalence of categories.
\end{theorem}

\begin{proof}
Given a nilpotent representation $$\xymatrix{V_- \ar@/^1.0pc/@[black][r]^{a} & V_+  \ar@/^1.0pc/@[black][l]^{b}}$$ one can recover the $D$-module $M$ by letting $\displaystyle M=\bigoplus_{k\le -1}V_- \oplus \bigoplus_{k\ge 0} V_+ $ and defining the actions of $x$ and $\partial$ in the following way:
$$
\xymatrix{\dots~~ \ar@/^1.0pc/@[black][r]^{ba-3}&
V_- \ar@/^1.0pc/@[black][r]^{ba-2}
\ar@/^1.0pc/@[black][l]^{1}
  & V_-   \ar@/^1.0pc/@[black][l]^{1} \ar@/^1.0pc/@[black][r]^{ba-1}
  & V_-  \ar@/^1.0pc/@[black][l]^{1} \ar@/^1.0pc/@[black][r]^{a}
  & V_+ \ar@/^1.0pc/@[black][l]^{b}
  \ar@/^1.0pc/@[black][r]^{ab+1}
  & V_+ \ar@/^1.0pc/@[black][l]^{1}
  \ar@/^1.0pc/@[black][r]^{ab+2} 
  & V_+ \ar@/^1.0pc/@[black][l]^{1}
  \ar@/^1.0pc/@[black][r]^{ab+3} 
  & ~\dots \ar@/^1.0pc/@[black][l]^{1}
  }
$$
One checks Weyl algebra relations are satisfied. For instance, $ 1 (ab+1) - a b = 1$ and $b a - (ba-1)1 = 1$.
\end{proof}

Let us discuss now the case of $D$-modules for $n>1$. That is consider a weight module $M$ over $D=\langle x_1, \partial_1, \dots, x_n,\partial_n\rangle$ with integer weights. Let $M=\bigoplus_{z\in\mathbb{Z}^n}M_z$ be its weight decomposition. By the same argument as for the case $n=1$, the action of $x_i$ establishes isomorphism between $M_{(z_1,\dots,z_i,\dots,z_n)}$ and $M_{(z_1,\dots,z_i+1,\dots,z_n)}$, unless $z_i=-1$. Therefore, every weight space is isomorphic to one of the $2^n$ weight space $M_z$, $z\in \{-1,0\}^n$. We can put these $2^n$ weight spaces at vertices of an $n$-dimensional cube, where each edge is replaced with two maps 
\xymatrix{ \ar@<0.5ex>[r]^{x_i} &   \ar@<0.5ex>[l]^{\partial_i}} . For instance, for $n=2$ and $n=3$ we get graphs
$$
\xymatrix{ M_{-1,0}\ar@<0.5ex>[r]^{x_1} \ar@<0.5ex>[d]^{\partial_2}&  M_{0,0}\ar@<0.5ex>[d]^{\partial_2} \ar@<0.5ex>[l]^{\partial_1}\\
 M_{-1,-1}\ar@<0.5ex>[r]^{x_1} \ar@<0.5ex>[u]^{x_2}& M_{0,-1}  \ar@<0.5ex>[l]^{\partial_1}\ar@<0.5ex>[u]^{x_2}}
 $$ % ~~~~~~ 
 and $$
 \xymatrix{& 
 M_{-1,0,0}\ar@<0.5ex>[dl]^{\partial_3}\ar@<0.5ex>[rr]^{x_1}\ar@<0.5ex>[dd]^(.7){\partial_2}|!{[dl];[dr]}\hole& & M_{0,0,0} \ar@<0.5ex>[ll]^{\partial_1}\ar@<0.5ex>[dd]^{\partial_2} \ar@<0.5ex>[dl]^{\partial_3}\\
 M_{-1,0,-1} \ar@<0.5ex>[ur]^{x_3}\ar@<0.5ex>[rr]^(0.7){x_1}\ar@<0.5ex>[dd]^{\partial_2}& & 
 M_{0,0,-1} \ar@<0.5ex>[ll]^(.65){\partial_1}\ar@<0.5ex>[ur]^{x_3}\ar@<0.5ex>[dd]^(0.7){\partial_2}\\& 
 M_{-1,-1,0} \ar@<0.5ex>[uu]^(.7){x_2}|!{[ur];[ul]}\hole\ar@<0.5ex>[rr]^(0.7){x_1}|!{[dr];[ur]}\hole \ar@<0.5ex>[dl]^{\partial_3}& & 
 M_{0,-1,0} \ar@<0.5ex>[ll]^(0.65){\partial_1}|!{[dl];[ul]}\hole \ar@<0.5ex>[dl]^{\partial_3} \ar@<0.5ex>[uu]^{x_2}\\
 M_{-1,-1,-1} \ar@<0.5ex>[ur]^{x_3}\ar@<0.5ex>[uu]^{x_2} \ar@<0.5ex>[rr]^{x_1}& & 
 M_{0,-1,-1} \ar@<0.5ex>[ur]^{x_3} \ar@<0.5ex>[uu]^(0.7){x_2} \ar@<0.5ex>[ll]^{\partial_1}}
$$

\input{tuesday3}
\input{tuesday4}
\newpage

\section{The Farkas lemma (Nick)}

\begin{definition}
  A polarized hyperplane arrangement is a triple $\calV = (V,\eta,\xi)$, where
  \begin{enumerate}
  \item $V$ is a subspace in $\bbR^I$,
  \item $\eta \in \bbR^I/V \cong (V^\perp)^*$, and
  \item $\xi\in V^* \cong (\bbR^I)^*/V^\perp$.
  \end{enumerate}
\end{definition}
For simplicity, we assume that $V$ is cut out by equations defined over $\bbZ$, and that $\eta$ and $\xi$ are also integral.
\todo{What exactly does ``integral'' mean?}

We have the following exact sequences.
Applying $\operatorname{Lie}(-)$ to the fourth sequence gives us the third sequence.
\begin{align*}
  0 &\from \bbR^I/V \from \bbR^I \from V \from 0\\
  0 &\to V^\perp \to (\bbR^I)^* \to V^*\to 0\\
  0 &\to V^\perp_\bbC \to (\bbC^I)^* \to V^*_{\bbC}\to 0\\
  1 &\to K \to (\bbC^\times)^I \to T \to 1.
\end{align*}

The subtorus $K\subset (\bbC^*)^I$ acts on $D = \bbC\langle x_i,\del_i \mid i\in I \rangle$.
Let $U = D^K$ be the invariants.
Then $Z(U) \cong \Sym \frakk = \Sym V^\perp_\bbC$.
We can think of $\eta$ as a map $\Sym V^\perp_\bbC\to \bbC$, or equivalently as a map $Z(U)\to \bbC$.
Set $U_\eta = U\otimes_{Z(U)}\bbC$.
Recall that we have
\[
  U = \bigoplus_{\lambda\in \frakt^*_\bbZ}D_0\cdot\frakm^\lambda,\quad U^{\geq 0} = \bigoplus_{\langle \lambda, \xi \rangle \geq 0}D_0\cdot \frakm^{\lambda}.
\]




 	

%%% Local Variables:
%%% TeX-master: "main"
%%% End:





\input{wednesday2}
\input{thursday1}
\input{thursday2}
\input{thursday3}
\input{thursday4}
\section{Friday: Alex}

\subsection{The non-abelian Higgs and Coulomb branches}

Plan 
\begin{itemize}
    \item Recap $\M_C,\M_H$
    \item See some parallels in structure/properties 
    \item Some examples
\end{itemize}

We start with a pair $(G,V)$ for $G$ a connected reductive group (over $\C$) and $V$ some finite dimensional $G$-rep. To this pair are associated a pair of varieties, $\M_H$ the Higgs branch, and $\M_C$ the Coulomb branch. 

Recall: Want to study $\OO_H, \OO_C$ and their duality properties. 

Notation: Denote by $T\subset G$ the maximal torus, by $W = N_G(T)/T$ the Weyl group, and by $\tt\subset \g$ the corresponding Lie algebras.

\subsection{The Higgs Branch}

Def: For $\eta : G\rightarrow \C^\times$ define 
\[
    \M_{H,\eta} = T^\ast V ////_\eta G    
\]
a Hamiltonian reduction.

What does this mean/unmasking. 

Hamiltonian $G$ action on $T^\ast V = V^\ast \oplus V$ with moment map $\mu : T^\ast V \rightarrow \g^\ast$ defined by $\mu(\phi,v) = (X\mapsto \phi(Xv))$ let's us define 
\[
    R = \bigoplus_{k\ge 0} R_k = \bigoplus_{k\ge 0} \C[\inv{\mu}(0)]^{G,k\eta} = \left(\C[\inv{\mu}(0)]\otimes \C[t] \right)^G
\]
where the superscript $G,k\eta$ means we are taking semi-invariants, and the action of $G$ on the grading is $g\cdot t = \inv{\eta(g)}t$. 

Note: Nick did this in the reverse in his talk yesterday. Imposed Lawrence relations after\ldots 

Then like in Nick's talk yesterday 
\[\M_{H,\eta} = \Proj R \rightarrow \M_{H,0} = \Spec R_{\eta,0} = \Spec \C[\inv{\mu}(0)]^G \]
Note $R$ depends on $\eta$ so maybe write $R_\eta = R$. 

Notes 
\begin{enumerate}
    \item If $\M_{H,\eta}$ is smooth, then $\M_{H,\eta} \rightarrow \M_{H,0}$ is a symplectic resolution. 
    \item From Justin's talk yesterday we know how to quantize such a variety. 
\end{enumerate}

Examples 
\begin{itemize}
    \item Hypertoric varieties 
    \item (Nakajima) quiver varieties
\end{itemize}

Remark: that's the Higg's branch. Easy to define compared with Coulomb. 

Remark: $G$ simple then $G$ has no characters. 

\subsection{The Coulomb side}

Consider $G((t)) = G(\C((t))), G[[t]],\ldots$. $G((t))$ acts on $V((t)) = V\otimes_{\C} \C((t))$. 

Define $\Xi_V = G((t)) \times^{G[[t]]} V[[t]] = (G((t))\times V[[t]])/G[[t]]$ have $\Xi_V\xrightarrow{\pi} V((t))$ defined by $[g,v]\mapsto gv$. Let ${}_V\Xi_V = \inv{\pi}(V[[t]])$. 
Another way to write this guy is 
\[
\cong \{
(P,\phi,s) \st \phi: P|_{D^\times}\xrightarrow{\cong} P_{triv}|_{D^\times}, s\in \Gamma(D,P\times^G V), \phi(s) \in \Gamma(D,P_{triv}\times V)    
\}    
\] 
where $P$ is a pgb on $D$. 

Def. $A = H_\bullet^{G[[t]]} ({}_V\Xi_V)$ and $A_{\hbar} = H_\bullet^{G[[t]]\rtimes\C^\times}({}_V\Xi_V)$. 

Note. $G[[t]]$ acts on ${}_V\Xi_V$ by $g_1\cdot [g_2,v] = [g_1g_2,v]$ and $\C^\times$ acts on $G((t))$ by $s\cdot g(t) = g(st)$. 

Morally, $$H_\bullet^{G[[t]]} ({}_V\Xi_V) = H_\bullet^{G((t))} (\Xi_V\times_{V((t))} \Xi_V) $$ 
(cf.\ $B\backslash G/B \cong G\backslash(G/B\times G/B)$.)

Theorem (BFN) $A, A_{\hbar}$ can be given associative ``convolution'' algebra structures. 
\begin{enumerate}
    \item $A$ is commutative 
    \item $A_{\hbar}$ is a deformation of $A$ over $\C[\hbar] = H^\bullet_{\C^\times} (\pt) $, i.e.\ $A_{\hbar}/\hbar A_{\hbar} \cong A$. In particular, this implies that $A$ has a Poisson structure. 
\end{enumerate}

Def. $\M_C = \Spec A$. 

Theorem (BFN) $A$ is f.g.\, a domain, and normal (i.e.\ all its local rings are integrally closed). This last property turns out to be very important when we try to compare $\M_c$ with some other space. It's also generically symplectic, but BFN don't say whether it's a symplectic singularity. So the Coulomb branch is always normal, but the Higgs branch is not neccessarily. There are examples of Coulomb branches which are not conical. [List.]

\subsubsection{Some properties} 

Completely general. 

\begin{enumerate}
    \item $H^\bullet_{G[[t]]}(\pt)\cong H^\bullet_G(\pt) \rightarrow A, A_{\hbar}$
    and the image is (Poisson) commutative. Note, this is analgous to the map $D_0\rightarrow U$ that we have seen in Ben's talk. Taking spec yields a map $\M_C\rightarrow \Spec H_G^\bullet(\pt) \cong \tt/W$ and the fact that the guy on the right is commutative means this is an integrable system. Moreover, $A$ is free over $H_G^\bullet (\pt)$. 
    \item Have a map \[\xymatrix{\M_C \ar@{-->}[rr]\ar[dr]& & T^\ast T^\vee /W\cong (\tt\times T^\vee)/W \ar[dl] \\ & \tt/W & }\] it implies $\dim \M_C = 2\rk G$ and $\M_C$ is generically symplectic. The dashed map is a birational symplectomorphism. Algebraically, this reflects the localization theorem. 
    \[\xymatrix{
        A(G,V) \ar[r] & A(T,V)[{\textrm{roots}}^{-1}]^W \ar[r] & A(T,\OO)[{\textrm{roots}}^{-1}]^W
    }\] with each map an embedding of algebras. Note, we take $T^\vee$ because we want functions on smth, so need to take coweights. 
\end{enumerate}

Examples
\begin{enumerate}
    \item Hypertoric varieties! (case $G = T\rightarrow (\C^\times)^n \subset GL(V)$ if $V\cong \C^n$)
    \item In general, if $G$ acts on $V = \OO$ then $\M_C\cong \Z_\g^\vee$ the scheme of regular centralizers $ = T^\ast G^\vee ////_{\rho,\rho}(U^\vee\times U^\vee)$ ``Kostant-Whittaker reduction.'' This guy is sometimes called ``pure $G$-gauge theory.'' Note $U^\vee$ is the unipotent radical of a Borel. 
    \item ``Adjoint matter'' got by $G$ acting on $V = \g$. $$\M_C\cong T^\ast T^\vee/W$$ Here $A_\hbar$ is a spherical graded Cherednik algebra if $G$ is in type $A$, maybe. 
    \item Quiver gauge theory supplies a family of examples. Start with a quiver $\Gamma$. For simplicity assume it is an oriented finite ADE dynkin diagram. Eg, $A_4$. Choose vector spaces $V_i,W_i$ for every vertex $i\in \Gamma$. Let
    $
    \mathbf{V} = \bigoplus_{i\rightarrow j} \hom(V_i,V_j)\oplus \bigoplus_{i}\hom(W_i,V_i)$ and $\mathbf{G} = \prod GL (V_i)$ and $\eta: \mathbf{G}\rightarrow \C^\ast$ defined by $(g_i)\mapsto \prod \det\inv{g_i}$. Then $\M_{H,\eta}$ is the Nakajima quiver variety. The dimension vectors $(\dim W_i)$ and $(\dim V_i)$ can be used to define coweights $\lambda,\mu$ for the (adjoint) ADE group. 

    Theorem (BFN et al.) \begin{enumerate}
        \item $\M_C\cong \W_\mu^\lambda$ a generalized slice in $\G$ for the ADE group 
        \item $A_{\hbar} = $ truncated shifted Yangian 
    \end{enumerate}

    Examples: \begin{enumerate}
        \item $\Gamma = \sl_2$ Dynkin $\rightsquigarrow W = \C^n\rightarrow V = \C^k$ and $\mathbf{G} = GL(k)$ acting on $\hom(\C^n,\C^k)$ and then \begin{enumerate}
            \item $\M_{H,\eta}\cong T^\ast Gr(k,n)$. 
            \item $\M_C= \W^k_{n-2k}$ for $$\M_C = \left\{  
                \begin{pmatrix} a(t) & b(t) \\ c(t) & d(t) \end{pmatrix} \in M_2 \C[t] \st \begin{aligned} a \textrm{ monic, degree }k\,, \\ b,c \textrm{ deg }<k\,, \\ \det = t^n \end{aligned}   
            \right\} $$ 
            Both varieties have connections to $\Rep \sl_2$ via the weight space $V(n)_{n-2k}\subset K_\theta (\OO)\subset V(1)^{\otimes n}$. $\OO\cong$ modules over $\otimes$-product algebra for $\sl_2$.  
        \end{enumerate}  
        \item When taking $V_i = \C^i$ up to $i = n-1$ and $W_{n-1} = \C^n$ but other $W = 0$ get $\M_{H,\eta} \cong T^\ast \fl_n$ and $\M_C\cong \mathcal{N}_{\sl_n}$ which $\M_{H,\eta}$ resolves. 
    \end{enumerate}
\end{enumerate} 

\section{Friday: Ben}

We take $\tt\, (\cong \tt^\ast $ Mon--Thur$)$ and draw in it all the hyperplanes $h_i^{mid} = n\,,\forall n\in\Z$. 

[Picture.] 

On this lattice I have morphisms between points. In one direction I have paths, in the other direction I have translations by $\tt_\Z$. Call the quiver $\M^\lambda$.
%  = \xymatrix{\dot \ar/_/[r] & \dot \ar@{/_/}[l]}$

The punchline, theorem: $\End(\OO_{+\epsilon})$ is Morita equivalent to $A_\hbar$ for $\M_C$. 



\input{friday2}
\input{friday3}
\input{friday4}
\newpage
\doclicenseThis
\end{document}
