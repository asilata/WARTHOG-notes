


\newpage

\section{Hypertoric Enveloping Algebra. Part I (Nick)}

\subsection{Weyl algebra}

Once and for all, fix a positive integer $n\ge1$. Consider the Weyl algebra
\begin{equation}
\begin{array}{cc}
    D = \mathbb{C}\langle x_1,\partial_1, x_2, \partial_2, \dots, x_n, \partial_n \rangle = \\ \{\textrm{polynomial differential operators on  } \mathbb{C}[x_1,x_2, \dots, x_n]\}
\end{array}
\end{equation}

As usual, the algebra $D$ has a filtration given by the order of the differential operators. However, we prefer to consider a different filtration, adjusted to the fact that we are considering differential operators on the affine space $\mathbb{C}^n$. Specifically, for $k\ge0$ let
\begin{equation}
F_k D = \Big\{ p_{k}(x_1,\partial_1, \dots, x_n, \partial_n)\Big| ~ p_{k} \textrm{ is a polynomial of degree at most } k \Big\}.
\end{equation}
\begin{example} 
$x_1\partial_1\in F_2 D$, $1\in F_0D$, $x_1^3\partial_2^2\in F_5D$.
\end{example}
Note that the associated graded of the filtration is the polynomial ring:
\begin{equation}
\displaystyle \gr D = \bigoplus_{k\ge0} F_k D/{F_{k-1}D} \cong \mathbb{C} [x_1,y_1,\dots, x_n, y_n],
\end{equation}
where $x_i$ denotes (the image of) multiplication by $x_i$, and $y_i$ denotes (the image of) $\partial_i$.

Consider the action $(\mathbb{C}^*)^n \acts D$ given by $t\cdot x_i = t_ix_i$, $t\cdot \partial_i = t_i^{-1}\partial_i$ for $t=(t_1,\dots,t_n)\in (\mathbb{C}^*)^n$.

For any $z\in \mathbb{Z}^n$, define the weight space $D_z=\{a\in D | t\cdot a = t^z a, \forall t \in (\mathbb{C}^*)^n\}$, where we use notation $(t_1,\dots,t_n)^{(z_1,\dots, z_n)} = t_1^{z_1}\dots t_n^{z_n}$. As usual, one has the decomposition $D=\bigoplus_{z\in\mathbb{Z}^n}D_z$.

\begin{example}
$D_0$ contains the elements $x_i\partial_i$ and $\partial_ix_i$, for any $i$. There elements are going to be of extreme importance for us. So let us fix notations for them. Let $h_i^+=x_i\partial_i$, $h_i^-=\partial_ix_i$. 
\end{example}
Note that $h_i^-=h_i^++1$. So, these two elements carry more or less the same information. At this point, it looks like it would be more convenient to switch $+$ with $-$ in the notation. However, this particular choice of notation will become convenient later on.

Note that any polynomial expression of $h_i^+$'s (as well as $h_i^-$'s) will represent an element of $D_0$. So, it is immediate that
$$
D_0 \supset \mathbb{C}[h_1^+,\dots,h_n^+] = \mathbb{C}[h_1^-,\dots, h_n^-].
$$
What is less obvious is that in fact the inclusion above is an equality. We are going to prove it next.

To prove that, we need to understand how to write down an expression of the sort $x_1\partial_1^3x_1x_2^4\partial_2^4 \in D_0$ as a polynomial applied to $(h_1^+,\dots, h_n^+)$.

\begin{lemma}\label{lm:action_of_h_on_D_0}
If $a\in D_0$, then $[h_i^\pm,a]=z_ia$. 
\end{lemma}
Note that it does not matter, which of $h_i^+$, $h_i^-$ we use in the formula, because they differ by a constant.
\begin{proof}

\end{proof}

\begin{lemma}\label{lm:express_as_prod_of_hs}
For any $k\ge0$ one has,
$$
\partial_i^kx_i^k=(h_i^+ + 1)(h_i^+ + 2)\dots (h_i^+ + k),
$$
$$
x_i^k\partial_i^k=(h_i^- - 1)(h_i^- - 2)\dots (h_i^- - k).
$$
\end{lemma}
\begin{proof}

\end{proof}

Lemma \ref{lm:express_as_prod_of_hs} gives an efficient way of expressing any element in $D_0$ as a polynomial in $h_i^\pm$'s.
For instance,

\begin{equation*}
\begin{array}{cc}
     x_1\partial_1^3x_1x_2^4\partial_2^4 = (x_1^2\partial_1^2)(\partial_1x_1)(x_2^4\partial_2^4)= \\
          =
    (h_1^--1)(h_1^--2)(h_1^++1)(h_2^--1)(h_2^--2)(h_2^--3)(h_2^--4).
\end{array}
\end{equation*}

\begin{corollary}
$D_0 = \mathbb{C}[h_1^+,\dots,h_n^+] = \mathbb{C}[h_1^-,\dots, h_n^-].$
\end{corollary}

Now that we have described $D_0$, let us discuss how to describe $D_z$ for $z\not=0$.

For any $z\in \mathbb{Z}^n$, denote by $z_+$ and $z_-$ the vectors with coordinates $(z_+)_i=\max(z_i,0)$, $(z_-)_i=\max(-z_i,0)$. Note that $z=z_+-z_-$. Denote $m^z=x^{z_+}\partial^{z_-}\in D_z$.

\begin{example}
If $z=(5,-2,0,1,1)$, then 
\[
m^z=x^{(5,0,0,1,1)}\partial^{(0,2,0,0,0)}=
x_1^5x_4x_5\partial_2^2\,.
\]
\end{example}

For any $f(h)=f(h_1^+, \dots, h_n^+)\in D_0$ and any $z\in \mathbb{Z}^n$, let $$f_z(h) = f(h_1^++z_1,\dots,h_n^++z_n).$$

\begin{lemma}\label{lm:commuting_poly_and_m}
$f(h)m^z = m^z f_z(h)$
\end{lemma}
\begin{proof}

\end{proof}

\begin{corollary} \label{cor:description_of_D_z}
For any $z\in \mathbb{Z}^n$, $D_z=\mathbb{C}[h_1^\pm,\dots,h_n^\pm]\cdot m^z$.
\end{corollary}
\begin{proof}
Follows from Lemmas \ref{lm:express_as_prod_of_hs}, \ref{lm:commuting_poly_and_m}.
\end{proof}

\begin{example}
Let $n=1$. Let us represent the element $x_1^8\partial_1^5\in D_3$ in the form described in Corollary \ref{cor:description_of_D_z}:

\begin{equation*}
    \begin{array}{ll}
         x_1^8\partial_1^5 = x_1^3(x_1^5\partial_1^5) &= 
         x_1^3(h_1^--1)(h_1^--2)\dots(h_1^--5) =\\
         &=(h_1^--4)(h_1^--5)\dots(h_1^--8)x_1^3.
    \end{array}
\end{equation*}
\end{example}

Corollary \ref{cor:description_of_D_z} provides the following description of $D$ (as a vector space):
$$
D=\bigoplus_{z\in\mathbb{Z}^n}D_z = \bigoplus_{z\in\mathbb{Z}^n}\mathbb{C}[h_1^\pm,\dots,h_n^\pm]\cdot m^z
$$
To understand the algebra structure of $D$ in terms of this decomposition, we need a formula for computing the product $f(h)m^z \cdot g(h)m^w$, for polynomials $f,g$ and $z,w\in\mathbb{Z}^n$. Lemma \ref{lm:express_as_prod_of_hs} guarantees that
$$
f(h)m^z \cdot g(h)m^w = f(h)g_z(h)m^z\cdot m^w,
$$
so reduces the problem to computing the product $m^z\cdot m^w$. Note that the latter apriori belongs to $D_{z+w}$, however there is no reason to expect it to be $m^{z+w}$. In fact, the following is true.

\begin{proposition} Let 

\begin{equation*}
    \begin{array}{ccc}
         [h_i]^k=&\left\{
         \begin{array}{lll}
              1, & \textrm{if } k=0,\\
              x_i^k\partial_i^k, & \textrm{if } k>0,\\
              \partial^{-k}x_i^{-k}, & \textrm{if } k<0,
         \end{array}
         \right.
         &
         =\left\{
         \begin{array}{lll}
              1, & \textrm{if } k=0,\\
              (h_i^+ + 1)\dots (h_i^+ + k), & \textrm{if } k>0,\\
              (h_i^- + 1)\dots (h_i^- + |k|), & \textrm{if } k<0.
         \end{array}
         \right.
    \end{array}
\end{equation*}

Then 

$$
    m^z\cdot m^w = \left(\prod_{\substack{z_iw_i<0 \\ |z_i|\leq |w_i|}}[h_i]^{z_i}\right)m^{z+w}\left(\prod_{\substack{z_iw_i<0 \\ |z_i|> |w_i|}}[h_i]^{w_i}\right).
$$
\end{proposition}
\begin{proof}
By example (in exercises).
\end{proof}

Note that each $z_i$ has the same sign as $w_i$, then the naive expectation will hold: $m^z\cdot m^w=m^{z+w}$.

\subsection{Hypertoric Enveloping Algebra}

To construct hypertoric enveloping algebra, we need to fix a subtorus $K\subset (\mathbb{C^*}^n)$. Throughout the notes we will assume the following genericity section:
\begin{align}\tag{S}\label{subtorus_condition}
    K \textrm{ does not contain any coordinate subtorus.}
\end{align}

Let $T=(\mathbb{C^*})^n/K$. Consider the following short exact sequences:

\begin{tikzcd}
    1\arrow{r} & K \arrow{r}& (\mathbb{C^*})^n \arrow{r}& T\arrow{r} & 1 \\ 
    0\arrow{r} & \mathfrak{k} \arrow{r}& \mathbb{C}^n \arrow{r}& \mathfrak{t}\arrow{r} & 0 \\
    0 & \arrow{l}\mathfrak{k} & \arrow{l}(\mathbb{C}^n)^* & \arrow{l}\mathfrak{t}^*& \arrow{l} 0 \\
    0 & \arrow{l}\mathfrak{k}_{\mathbb{Z}}\ar[equal]{d}& \arrow{l}\mathbb{Z}^n \ar[equal]{d}& \arrow{l}\mathfrak{t}^*_{\mathbb{Z}}\ar[equal]{d}& \arrow{l} 0 \\
     & \Hom(K,\mathbb{C}^*) & \Hom((\mathbb{C}^*)^n,\mathbb{C}^*) & \Hom(T,\mathbb{C}^*) &  \\
\end{tikzcd}

\begin{definition}
The hypertoric enveloping algebra (HEA, for short) is
$$
U=D^K = \bigoplus_{z\in \mathfrak{t}^*_{\mathbb{Z}}\subset\mathbb{Z}^n} \mathbb{C}[h_1^\pm,\dots,h_n^\pm]\cdot m^z.
$$
\end{definition}

\begin{example}\label{ex:diagonal_in_C3}
Let $n=3$, $K=\mathbb{C}^*_\Delta$ be the diagonal copy of $\mathbb{C}^*$ sitting inside $(\mathbb{C}^*)^3$. Then $T=(\mathbb{C}^*)^3/\mathbb{C}^*_\Delta\cong (\mathbb{C}^*)^2$. We have short exact sequences:

\begin{tikzcd}[row sep=tiny]
    1\arrow{r} & \mathbb{C}^* \arrow{r}& (\mathbb{C^*})^3 \arrow{r}& (\mathbb{C^*})^2\arrow{r} & 1 \\ 
    & t \ar[maps to]{r}&(t,t,t)&&\\
    &&(t_1,t_2,t_3)\ar[maps to]{r}& (t_1/t_3,t_2/t_3)&
\end{tikzcd}

\begin{tikzcd}[row sep=large]
    1\arrow{r} & \mathbb{C} \arrow{r}{\begin{bmatrix} 1  \\ 1 \\ 1 \end{bmatrix}} & [3em]\mathbb{C}^3 \arrow{r}{\begin{bmatrix} 1~&0&-1  \\ 0~&1&-1 \\ \end{bmatrix}}&[3em] \mathbb{C}^2\arrow{r} & 1 \\ 
        1 & \arrow{l}\mathbb{C} & \arrow{l}[swap]{\begin{bmatrix} 1&1&1 \end{bmatrix}}\mathbb{C}^3 & \arrow{l}[swap]{\begin{bmatrix} 1&~~0\\0&~~1 \\ -1&-1 \\ \end{bmatrix}} \mathbb{C}^2 &\arrow{l} 1 \\     1 & \arrow{l}\mathbb{Z} &                     \arrow{l}\mathbb{Z}^3 &  \arrow{l}\mathbb{Z}^2 &\arrow{l} 1 
\end{tikzcd}

\marginnote{picture here?}
\end{example}

The last question we will address in this section is: What is the center of $U$?

Recall that Lemma \ref{lm:action_of_h_on_D_0}, if $a\in D_0$, then $[h_i^+,a]=z_ia$. In particular, we get that $Z(U)\subset D_0$.

Also, recall that by Lemma \ref{lm:commuting_poly_and_m} $f(h)m^z=m^zf_z(h)$. So, typically $Z(U)$ is strictly smaller than $D_0$.  Moreover $f(h)$ is central iff $f(h)=f_z(h)$, for all $z\in\mathfrak{t}^*_{\mathbb{Z}}\subset \mathbb{Z}^n$.

\begin{proposition}\label{prop:center_of_HEA}
$Z(U)=\Sym(\mathfrak{k})\subset \Sym(\mathbb{C}^n)=\mathbb{C}[h_1^+,\dots,h_n^+].$
\end{proposition}
\begin{proof}

\end{proof}

\begin{example}
In the setting of Example \ref{ex:diagonal_in_C3}, $Z(U)=\mathbb{C}[h_1^++h_2^++h_3^+]$.
\end{example}

