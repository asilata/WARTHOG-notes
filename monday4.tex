\newpage

\section{Weight Modules (Ben)}

Recall that $D$ stands for the Weyl algebra of polynomial differential operators on $\mathbb{C}[x_1,\dots,x_n]$. If $K$ is a subtorus of $(\mathbb{C^*})^n$ (not containing any coordinate subtorus), then we define the hypertoric enveloping algebra (HEA) as $U=D^K$. The HEA contains an abelian subalgebra $D_0=\mathbb{C}[h_1^+,\dots,h_n^+$.

In this section we will describe simple weight modules of $U$. Throughout the section, by a $U$-module we mean a left $U$-module.

Let $M$ be a finite dimensional $U$-module. Let us decompose this module into eigenspaces of $h_i^+$'s.

\begin{definition}
Let $\chi$ be a surjective homomorphism $D_0\to \mathbb{C}$. Denote by $m_\chi$ the maximal ideal $\ker(\chi:D_0\to \mathbb{C})$. Define
$$
M_\chi = \{v\in M | m_\chi^N=0 \textrm{ for some } N\ge1\}.
$$
Alternatively, if we denote $\chi_i=\chi(h_i^+)$, then
$$
M_\chi = \{v\in M | (h_i^+-\chi_i)^N=0 \textrm{ for some } N\ge1\}.
$$
\end{definition}
\begin{definition}
A $U$-module $M$ is called a weight module if $\displaystyle M\cong \bigoplus_{\chi}M_\chi$.
\end{definition}


\begin{theorem}
If $M$ is a finite dimensional $U$-module, then it is a weight module.
\end{theorem}

\begin{proof}
For any $i$, consider the Jordan decomposition $h_i^+=h_i^{ss}+h_i^{n}$, where $h_i^{ss}$ acts on $M$ semi-simply and $h_i^n$ nilpotently. Recall that both $h_i^{ss}$ and $h_i^n$ are polynomials in $h_i^+$. In particular the operators $h_i^+, 1\le i\le n$ are simultaneously diagonalizable.
\end{proof}

\begin{remark}
The action of $h_i^n$ is a $U$-module homomorphism $M\to M$, and moreover it commutes with any homomorphism of $U$-modules $M\to N$.
\end{remark}

\begin{corollary}
If $L$ is a simple finite dimensional $U$-module, then each $h_i$ acts semi-simply.
\end{corollary}

\begin{proof}
The image of $h_i^n$ is a submodule, and it cannot be the whole $L$. So, $h_i^n$ acts trivially.
\end{proof}

For any (surjective homomorphism) $\chi:D_0\to \mathbb{C}$, define the module $U$-module $P^{(1)}_\chi=U/Um_\chi$.
\marginnote{do we need modules $P_\chi^{(N)}$?}
It is an easy observation that for any $U$-module $M$, one has $M_\chi=\Hom(P_\chi^{(1)}, M)$. In particular, if $L$ is a simple $U$-module  and $L_\chi\neq \emptyset$ for some $\chi$, then $L$ is a quotient of $P_\chi^{(1)}$.

Let $w \in P_\chi^{(1)}$ be image of $1$. Note that $w\in (P_\chi^{(1)})_\chi $.

\begin{theorem}
$P_\chi^{(1)}$ has a basis given by $\{m^\lambda \cdot w\}_{\lambda\in \mathfrak{t}^*_{\mathbb{Z}}}$.
\end{theorem}
\begin{proof}

\end{proof}

\begin{corollary}
If $m_\chi,m_\nu\in \Spec_m(D_0)$, then $\left(P_\chi^{(1)}\right)_\nu=\left\{\begin{matrix}1&,\chi-\nu \in \mathfrak{t}^*_\mathbb{Z},\\0&,\textrm{otherwise}. \end{matrix}\right.$ 
\end{corollary}

\begin{lemma}\label{lm:proper_submodule_criterion}
A $U$-submodule $N\subset P_\chi^{(1)}$ is proper iff $N_\chi=0$.
\end{lemma}

\begin{proof}
If $N_\chi\neq\emptyset$, then $N_\chi=\left(P_\chi^{(1)}\right)_\chi\ni w$, so $N=P_\chi^{(1)}$.
\end{proof}

\begin{theorem}
Each $P_\chi^{(1)}$ has a unique simple quotient.
\end{theorem}
\begin{proof}
Existence is guaranteed by Zorn's Lemma. Uniqueness follows from the fact that sum of two proper submodules of $P_\chi^{(1)}$ is again a proper submodule. The latter fact follows direcly from Lemma \ref{lm:proper_submodule_criterion}.
\end{proof}

Denote the unique simple quotient of $P_\chi^{(1)}$ by $S_\chi$.

\begin{remark}
The module $S_\chi$ is not necessarily finite dimensional, but it is a weight module, since $P_\chi^{(1)}$ is.
\end{remark}

\begin{proposition}
For any $m_\chi,m_\nu\in\Spec_m(D_0)$, the following are equivalent:
\begin{enumerate}
    \item $P_\chi^{(1)}\cong P_\nu^{(1)}$,
    \item $S_\chi\cong S_\nu$,
    \item $\left(S_\chi\right)_\nu \neq \emptyset$,
    \item $\dim \left(S_\chi\right)_\nu =1 $,
    \item $m^{\chi-\nu}m^{\nu-\chi}w\neq 0\in P_\chi^{(1)}$.
\end{enumerate}
\end{proposition}

\begin{proof}

\end{proof}

Let us define an equivalence relation on the space of characters $D_0\to \mathbb{C}$, by declaring $\chi \sim \nu$, if any of the five conditions above holds.

\begin{theorem}\label{thm:equivalence_of_simples}
Let $m_\chi,m_\nu\in\Spec_m(D_0)$. Then $\chi \sim \nu$ iff $\chi-\nu\in\mathfrak{t}^*_\mathbb{Z}$ and there is no $i$ such that $h_i^-(\chi)$ and $h_i^+(\nu)$ both are integers and have opposite signs.
\end{theorem}

\begin{example}
Let $n=2$, $K=\{1\}$, $\mathfrak{t}^*_\mathbb{Z}\cong \mathbb{Z}^2$. Get $4$ equivalence classes on $\mathfrak{t}^*_\mathbb{Z}$. \marginnote{need a picture here}
\end{example}

Fix $\xi \in \mathfrak{k}^*_\mathbb{Z}$, and look at the (lattice in) affine space $\xi+\mathfrak{t}^*_\mathbb{Z}$. Each function $h_i^\pm$ is integral either at all points of $\xi+\mathfrak{t}^*_\mathbb{Z}$, or at none of them. 

If each $h_i^\pm$ is integral at none of the points $\xi+\mathfrak{t}^*_\mathbb{Z}$, then the condition in Theorem \ref{thm:equivalence_of_simples} vacuously holds true, and so all the points in $\xi+\mathfrak{t}^*_\mathbb{Z}$ are equivalent. In particular, for a generic $\xi$, there is unique simple module with a non-zero weight space in $\xi+\mathfrak{t}^*_\mathbb{Z}$. This is given by $P_\chi^{(1)}$ for any $\chi \in \xi+\mathfrak{t}^*_\mathbb{Z}$.

If some $h_i^\pm$ are integral at all the points of $\xi+\mathfrak{t}^*_\mathbb{Z}$, then this affine lattice will be divided by the level sets $h_i^\pm=0$ into several chambers. The set of chambers containing at least one point will be in a bijection with the set of simple $U$-modules. Moreover, the chambers with finitely many points will correspond exactly to the finite dimensional $U$-modules.

